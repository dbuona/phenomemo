\documentclass{article}\usepackage[]{graphicx}\usepackage[]{color}
%% maxwidth is the original width if it is less than linewidth
%% otherwise use linewidth (to make sure the graphics do not exceed the margin)
\makeatletter
\def\maxwidth{ %
  \ifdim\Gin@nat@width>\linewidth
    \linewidth
  \else
    \Gin@nat@width
  \fi
}
\makeatother

\definecolor{fgcolor}{rgb}{0.345, 0.345, 0.345}
\newcommand{\hlnum}[1]{\textcolor[rgb]{0.686,0.059,0.569}{#1}}%
\newcommand{\hlstr}[1]{\textcolor[rgb]{0.192,0.494,0.8}{#1}}%
\newcommand{\hlcom}[1]{\textcolor[rgb]{0.678,0.584,0.686}{\textit{#1}}}%
\newcommand{\hlopt}[1]{\textcolor[rgb]{0,0,0}{#1}}%
\newcommand{\hlstd}[1]{\textcolor[rgb]{0.345,0.345,0.345}{#1}}%
\newcommand{\hlkwa}[1]{\textcolor[rgb]{0.161,0.373,0.58}{\textbf{#1}}}%
\newcommand{\hlkwb}[1]{\textcolor[rgb]{0.69,0.353,0.396}{#1}}%
\newcommand{\hlkwc}[1]{\textcolor[rgb]{0.333,0.667,0.333}{#1}}%
\newcommand{\hlkwd}[1]{\textcolor[rgb]{0.737,0.353,0.396}{\textbf{#1}}}%
\let\hlipl\hlkwb

\usepackage{framed}
\makeatletter
\newenvironment{kframe}{%
 \def\at@end@of@kframe{}%
 \ifinner\ifhmode%
  \def\at@end@of@kframe{\end{minipage}}%
  \begin{minipage}{\columnwidth}%
 \fi\fi%
 \def\FrameCommand##1{\hskip\@totalleftmargin \hskip-\fboxsep
 \colorbox{shadecolor}{##1}\hskip-\fboxsep
     % There is no \\@totalrightmargin, so:
     \hskip-\linewidth \hskip-\@totalleftmargin \hskip\columnwidth}%
 \MakeFramed {\advance\hsize-\width
   \@totalleftmargin\z@ \linewidth\hsize
   \@setminipage}}%
 {\par\unskip\endMakeFramed%
 \at@end@of@kframe}
\makeatother

\definecolor{shadecolor}{rgb}{.97, .97, .97}
\definecolor{messagecolor}{rgb}{0, 0, 0}
\definecolor{warningcolor}{rgb}{1, 0, 1}
\definecolor{errorcolor}{rgb}{1, 0, 0}
\newenvironment{knitrout}{}{} % an empty environment to be redefined in TeX

\usepackage{alltt}
\usepackage{Sweave}
\usepackage{float}
\usepackage{graphicx}
\usepackage{tabularx}
\usepackage{siunitx}
\usepackage{mdframed}
\usepackage{natbib}
\bibliographystyle{..//refs/styles/besjournals.bst}
\usepackage[small]{caption}
\setkeys{Gin}{width=0.8\textwidth}
\setlength{\captionmargin}{30pt}
\setlength{\abovecaptionskip}{0pt}
\setlength{\belowcaptionskip}{10pt}
\topmargin -1.5cm        
\oddsidemargin -0.04cm   
\evensidemargin -0.04cm
\textwidth 16.59cm
\textheight 21.94cm 
%\pagestyle{empty} %comment if want page numbers
\parskip 0pt
\renewcommand{\baselinestretch}{2}
\parindent 15pt

\newmdenv[
  topline=true,
  bottomline=true,
  skipabove=\topsep,
  skipbelow=\topsep
]{siderules}
\IfFileExists{upquote.sty}{\usepackage{upquote}}{}
\begin{document}
\title{Life History Theory and Floral-Foliate Phenological Patterns in Temperate Forest Trees}
\author{Daniel Buonaiuto}
Daniel Buonaiuto
\par OEB 53
\par\data{\today}

Green is the color of spring, but any keen observer walking the temperate, deciduous forest of the Eastern United States early in the season would readily witness that it is often the subtle whites, reds and yellows of emerging tree flowers that are the first harbingers of spring in temperate forest communities. In some deciduous tree species, seasonal flowering proceeds leaf development, while in others, it is leaf expansion that occurs first. The study of phenology, the timing of annual life cycle events, has a long history, and even in the late 1800's, naturalists speculated that such contrasting floral-foliate sequences were not merely incidental, but that these patterns, in and of themselves, may be adaptive \citep{}. However, despite increasing scientific interest in the study of phenology over the past several decades, the phenology of reproductive (flowering, fruiting) and productive (budbust, leafout and drop) stages have long been treated separately, and both the mechanisms and effects of floral-foliate phenological patterns remain poorly studied empirically \citep{Wolkovich2014}.
\par Even finding suitable language to describe floral-foliate patterns in the existing listerature is an difficult endevour. Early botanical dictionaries define flowering followed by leaves as both "proterany" and "hysteranthy" (which gramaticaly should be antonyms). Other describe flowering before leafing as "precocious" flower, but that term can also refer to flowering early in ontogeny and have nothing do do with seasonality. To the aim of maintaining a consistancy of usage, I will adpot the terminology used by Auth \citeyear{} in which proteranthy refers to flowering before leafing, synanthy refers to flowering and leafing simaltaneously and  seranthy refers to flowering after leafing.
\par As global climate is predicted to change dramatically in the commoing decades It is imperative that we, as scientists, better understand these phenological patterns . The effects of climate change have already appeared in phenology \citep{menzel, wolkovich} and the degree to which these phenological shifts are altering floral foliate sequences is virtually unknown. If the sequences themselves are indeed adaptive, confering a signifcant fitness benefit to indivudals under historical conditions, disruptions and alterantions to these patterns cause by changing climate conditons could have negative demographi consiquences for many forest trees. To better understand the importance of these sequences and the ability for species to maintain them in a changing world, researchers should focus their attention on gaining a more complete picture of mechansims and effects of this. To this end, in section one of this paper, I will first present the dominant hypothesis for proteranthy in the context of life history theory, and them evaluate the empircal and theoretcal evidence for its support. In section two, I will discuss some of the biological mechansims producing the phenological patterns we see today and discuss how they may enable or constrain plastic responses to changing climate in forest trees.
\section{Proteranthy and Life History Theory}
\par Life history theory seeks to explain how organisms acheive reproductive success. The classical theory is based on an optimization model- life history traits of organisms (for example: age of reproduction, seed size ) are determined by tradeoffs in both extrinsic (environmental, community) and intrinsic (genetics, physiology) factors, which result in a lineage specific optimum for life history characters \citep{Stearns2000}. Typically, life history theory is applied to the full lifespan of an organism, but in the context of tree phenology, I will consider its optimumization model in the context of seasonal optimization. 

\par For flowering alone, optimization in a seasonal environment depends on several factors. For flower tissues and ultimately reprodcutive output, there is likely tradeoff between flowering minimizing risk for early season frost damaage and maintaining enough time for fruit development and dispersal. The timing is further selection by the vectors of pollination.  There is also an observed tradeoff where there is increased pollinator abundance for midseason flowers, but also increased competition to attract them \citep{}.There is likely a tradeoff between pollination effeciency and investment which I will discuss further below. Now, considering leaf phenology alone, optimization is though to maximize the growing season and minimizing the risk of damange from late season frost \citep{}. 
\par But with spring conditions varying greatly interannual how do decidious trees 

But now we must consider the timing of leaves and flowers together. Might the presence of leaves change the behaviors of pollen vectors? Might the presences of flowers with out leaves change the resource allocation dynamics? The sequencing of leaves and flowers, in and of itself, produces its own set of tradeoffs, which I will now discuss as we review the main hypothesis about proteranthy.
\par Proteranthy is thought to be an adaptation for pollination efficiency. Theorists explain that this trait is common in wind pollinated species, because producing flowers in the leafless state allows for maximum wind flow through the canopy and sigificantly reduces the potential for pollen interception by non floral parts \citep{}. While usually assumed associated in the literature with wind pollination, similary theory could be applied to insect pollinated species in that tree flowers are easier for pollinators to located when there are no leaves as barriers or obsticles.  Presumably, more effecient pollination would allow for species to reduce their overall investment in reproduction. However, their would still be costs associated with this life history trait. Proteranthous flowering would only be effective if it occured before the community as a whole leafed out, which would push such flowering early into the season to a time when risk of frost damage is high. Additionally, proteranthous flowering probably has an energetic cost, taking place at a time of the year when stored carbohydrates are at their lowest, with out the assistance of supplimental carbon from foliar photosythesis\citep{}. To my knowledge, there have been no empircal studies testing this hypothesis, but several studies seem to support it though indirect evidence.
Independence vs. constrait.
Phenophases are not optimized in a vaccum, but timing is based on that leaf and flower physiology and the functional relationship between them.

evidence: Wind pollination arose at same time of decidiousness, modeling wind flow through canopy, interception in that bog paper. dogwoods

\par We understand that phenophases are not optimized in a vaccum, but timing is based on that leaf and flower physiology and the functional relationship between them. Climate change is already having dramatic impact on phenology. cite lizzie. Will flowering and leafing phenophase shift relative to each other, maintaining their optimized temporal relationship?
To what degree is their timing constrained
Lechowitz
respond to same cues
same genetic pathways?

same cues ( cherry paper)
my work




\end{document}
scrap:After briefly considering the selective pressures acting on the timing floral and foliate stages (henceforth: phenophases) in the context of life history theory seperately, I will then present a framework for considering the relationships between floral and foliate phenophases and evaluating the significance of these temporal sequences.
