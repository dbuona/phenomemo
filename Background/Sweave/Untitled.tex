\documentclass{article}\usepackage[]{graphicx}\usepackage[]{color}
%% maxwidth is the original width if it is less than linewidth
%% otherwise use linewidth (to make sure the graphics do not exceed the margin)
\makeatletter
\def\maxwidth{ %
  \ifdim\Gin@nat@width>\linewidth
    \linewidth
  \else
    \Gin@nat@width
  \fi
}
\makeatother

\definecolor{fgcolor}{rgb}{0.345, 0.345, 0.345}
\newcommand{\hlnum}[1]{\textcolor[rgb]{0.686,0.059,0.569}{#1}}%
\newcommand{\hlstr}[1]{\textcolor[rgb]{0.192,0.494,0.8}{#1}}%
\newcommand{\hlcom}[1]{\textcolor[rgb]{0.678,0.584,0.686}{\textit{#1}}}%
\newcommand{\hlopt}[1]{\textcolor[rgb]{0,0,0}{#1}}%
\newcommand{\hlstd}[1]{\textcolor[rgb]{0.345,0.345,0.345}{#1}}%
\newcommand{\hlkwa}[1]{\textcolor[rgb]{0.161,0.373,0.58}{\textbf{#1}}}%
\newcommand{\hlkwb}[1]{\textcolor[rgb]{0.69,0.353,0.396}{#1}}%
\newcommand{\hlkwc}[1]{\textcolor[rgb]{0.333,0.667,0.333}{#1}}%
\newcommand{\hlkwd}[1]{\textcolor[rgb]{0.737,0.353,0.396}{\textbf{#1}}}%
\let\hlipl\hlkwb

\usepackage{framed}
\makeatletter
\newenvironment{kframe}{%
 \def\at@end@of@kframe{}%
 \ifinner\ifhmode%
  \def\at@end@of@kframe{\end{minipage}}%
  \begin{minipage}{\columnwidth}%
 \fi\fi%
 \def\FrameCommand##1{\hskip\@totalleftmargin \hskip-\fboxsep
 \colorbox{shadecolor}{##1}\hskip-\fboxsep
     % There is no \\@totalrightmargin, so:
     \hskip-\linewidth \hskip-\@totalleftmargin \hskip\columnwidth}%
 \MakeFramed {\advance\hsize-\width
   \@totalleftmargin\z@ \linewidth\hsize
   \@setminipage}}%
 {\par\unskip\endMakeFramed%
 \at@end@of@kframe}
\makeatother

\definecolor{shadecolor}{rgb}{.97, .97, .97}
\definecolor{messagecolor}{rgb}{0, 0, 0}
\definecolor{warningcolor}{rgb}{1, 0, 1}
\definecolor{errorcolor}{rgb}{1, 0, 0}
\newenvironment{knitrout}{}{} % an empty environment to be redefined in TeX

\usepackage{alltt}
\usepackage{Sweave}
\usepackage{float}
\usepackage{graphicx}
\usepackage{tabularx}
\usepackage{siunitx}
\usepackage{mdframed}
\usepackage{natbib}
\bibliographystyle{..//refs/styles/besjournals.bst}
\usepackage[small]{caption}
\setkeys{Gin}{width=0.8\textwidth}
\setlength{\captionmargin}{30pt}
\setlength{\abovecaptionskip}{0pt}
\setlength{\belowcaptionskip}{10pt}
\topmargin -1.5cm        
\oddsidemargin -0.04cm   
\evensidemargin -0.04cm
\textwidth 16.59cm
\textheight 21.94cm 
%\pagestyle{empty} %comment if want page numbers
\parskip 0pt
\renewcommand{\baselinestretch}{2}
\parindent 15pt

\newmdenv[
  topline=true,
  bottomline=true,
  skipabove=\topsep,
  skipbelow=\topsep
]{siderules}
\IfFileExists{upquote.sty}{\usepackage{upquote}}{}
\begin{document}
\title{Life History Theory and Floral-Foliate Phenological Patterns in Temperate Forest Trees}
\author{Daniel Buonaiuto}
Daniel Buonaiuto
\par OEB 53
\par\data{\today}

Green is the color of spring, but any keen observer walking the temperate, deciduous forest of the Eastern United States early in the season would readily witness that it is often the subtle whites, reds and yellows of emerging tree flowers that are the first harbingers of spring in temperate forest communities. In some deciduous tree species, seasonal flowering proceeds leaf development, while in others, it is leaf expansion that occurs first. The study of phenology, the timing of annual life cycle events, has a long history, and even in the late 1800's, naturalists speculated that such contrasting floral-foliate sequences were not merely incidental, but that these patterns, in and of themselves, may be adaptive \citep{Robertson1885}. However, despite increasing scientific interest in the study of phenology over the past several decades, the phenology of reproductive (flowering, fruiting) and productive (bud burst, leaf out and drop) stages have long been treated separately, and both the mechanisms and effects of floral-foliate phenological patterns remain poorly studied \citep{Wolkovich2014}.
\par Even finding suitable language to describe floral-foliate patterns in the existing literature is an difficult endeavor. Early botanical dictionaries define the pattern of flowering preceding leafing as both "proteranthy" and "hysteranthy" (which grammatically should be antonyms). Others describe flowering before leafing as "precocious" flowering, but this term can also refer to flowering early in ontogeny and may have nothing to do with seasonality. To the aim of maintaining a consistency of usage, I will adopt the terminology used by Lamont and Downes \citeyear{Lamont2011} in which \textit{proteranthy} refers to flowering before leafing, \textit{synanthy} refers to flowering and leafing simultaneously, and \textit{seranthy} refers to flowering after leafing.
\par As environmental conditions are predicted to dramatically change significantly in the comping decades due to anthropogenic climate change, it is imperative that we, as scientists, better understand these phenological patterns.The effects of climate change are already influencing the phenologies of a diversity of taxa, including trees \citep{Menzel2006}, and the degree to which these phenological shifts are altering floral-foliate sequences is virtually unknown. If the sequences themselves are indeed adaptive, conferring a significant fitness benefit to individuals under historical conditions, disruptions to these patterns caused by changing climate conditions could have negative demographic consequences for many forest tree species. To better understand the importance of these sequences and the ability for species to maintain or beneficially adjust them in a changing world, researchers should focus their attention on gaining a more complete picture of mechanisms dictating these patterns as well as their effects on the reproductive success and other life history traits of deciduous trees. To this end, in section one of this paper, I will first present the dominant hypothesis for proteranthy in the context of life history theory, and then evaluate the empirical and theoretical evidence for its support. In section two, I will discuss some of the biological mechanisms producing the phenological patterns we see today and discuss how they may enable or constrain plastic responses of forest trees to changing climate.
\section{Proteranthy and Life History Theory}
\par Life history theory seeks to explain how organisms achieve reproductive success. The classical theory is based on an optimization model-- life history traits of organisms (for example: age of reproduction, seed size) are determined by trade offs in both extrinsic (environmental, community) and intrinsic (genetics, physiology) factors, which result in a lineage specific optimum for life history characters \citep{Stearns2000}. Typically, life history theory is applied to the full lifespan of an organism, and much of the work in plants has investigated the factors that determine the transition between vegetative growth and reproductive life stages \citep{Glover2014}. But for trees, being long lived, perennial organisms, these transitions are considerably less discrete, and the interplay between vegetative and reproductive development is far more "fuzzy", repeating annually for much of the organisms' lifetimes. For this reason, classical life history theory is certainly applicable, but there may be some key differences from what we know about the life cycles of annual plants and seasonal phenology in long lived trees. With that caveat, I will attempt to apply the life history theory model of reproductive optimization in a seasonal context.
\par Before I go on, we should consider the environmental conditions that typically induce phenological responses. For plants in general, both genetic adaptation and phenotypic plasticity are thought to be involved in determining phenological patterns, and the partitioning of these influences has only recently begun to be explored \citep{Anderson2012}. Because trees are long lived organisms, with generation times that often exceed the duration of already observed phenological shifts, it is likely that the phenological shifts we have seen thus far as a reaction to climate change are plastic in nature. For temperate woody plants, it is generally accepted that the dominant cues for spring phenological events such as flowering and leaf out are vernalization temperatures (chilling in winter), forcing temperatures (warming in spring) and photoperiod, but it is clear that the interactions between these cues are complex, and behave differently for different species and in different locations \citep{Forrest2010}. Optimization of phenological timing depends on how trees accurately "interpret" these cues as reliable signals of overall seasonal patterns. For example, warm forcing temperatures are only reliable cues if they tend to coincide the with onset of spring. As such, periodic warm spells in the heart of winter could "fool" plants into initiating phenological events in a sub-optimal season. 
\par With this in mind, I will introduce a key metric for evaluating phenological responses to changing climatic conditions. Phenological sensitivity is a measure of the change in phenological event day per unit change in environmental condition (degrees C for temperature and hours for photoperiod). Because environmental conditions vary considerably over species' ranges, it is assumed that trees have locally adapted different sensitivities to these cues combinations which are well matched to the greater climate patterns in their specific growing environment. Selection maintains these differential sensitivities, as genotypes that "interpret" their environment more successfully will ultimately have a higher lifetime fitness. We will come back to the topic of sensitivity in detail in part two of the paper when we discuss the biological mechanisms that are responsible for the phenological patterns we observe in nature today.
\par For flowering alone, optimization in a seasonal environment depends on several evolutionary drivers. For flowers, and ultimately reproductive output, there is likely tradeoff between minimizing risk for early season frost damage and extending the time allotted to fruit development and propagule dispersal. Flowering timing is further selected upon by vectors of pollination and dispersal, and interactions with antagonists \citep{Austen2017}. These factors may interact in complex ways, and the strength of each driver might vary considerably for place to place and year to year, highlighting the importance of plasticity in floral phenological responses. Considering leaf phenology alone, optimization is thought to be a tradeoff between maximizing the growing season and minimizing the risk of damage from late season frost \citep{Kramer1995}, though interactions with symbiots could also affect the timing of leaf phenology. 
\par But now we must consider the timing of leaves and flowers together. Might the presence of leaves change the behaviors of pollination vectors? Might the presences of flowers without leaves change a tree species' resource allocation dynamics? The sequencing of leaves and flowers, in and of itself, produces its own set of tradeoffs, which I will now discuss as we review the main hypothesis about proteranthy.
\par Proteranthy is thought to be an adaptation for pollination efficiency. Theorists explain that this trait is common in wind pollinated species, because producing flowers in the leafless state allows for maximum wind flow through the canopy and significantly reduces the potential for pollen interception by non-floral structures \citep{Rathcke1985, Whitehead1969}.Though proteranthy is often discussed in the context of wind pollination, similar theory could be applied to insect pollinated species in that tree flowers are easier for pollinators to located when there are no leaves serving as detection barriers or physical obstacles. Presumably, more efficient pollination would allow for species to reduce their overall investment in reproduction. However, there would still be costs associated with this life history trait. Proteranthous flowering would only be effective if it occurred before the community as a whole leafed out, which would push such flowering early into the season to a time when risk of frost damage is elevated. Additionally, proteranthous flowering probably has an energetic cost, taking place at a time of the year when stored carbohydrates are at their lowest, without the assistance of supplemental carbon from foliar photosynthesis\citep{Aschan2003}.
To my knowledge, there have been no empirical studies testing the fitness benefits of proteranthy, but several studies seem to support it indirectly. There is evidence that wind pollination, a derived trait in angiosperms, arose at the same time as decidiousness \citep{Whitehead1969}. While this fact doesn't address proteranthy directly, it can be argued that this coincidence indicates that a leafless season is a necessary condition for wind pollination. This evolutionary argument can be supplemented by an observation from biogeography that wind pollination is rarely found in the tropics (where most woody species are evergreen), and common in the temperate and boreal zones where there is a distinct leafless season \citep{Whitehead1969}.
\par Other studies have more directly measured changes in pollen interception by non-floral plants structures at different stages of canopy closure. \citep {Tauber1967, Milleron2012}, and found strong evidence for increased filtration of pollen by non-reproductive structures as canopy closure progressed. These studies provide evidence that canopy fill does create a significant barrier to pollen transfer in forest, but they fall short of concretely supporting the adaptive significance of proteranthy because they are not able to quantify the direct effects of pollen filtration on tree fitness.
\par Another approach to obtain indirect evidence of a fitness benefit of proteranthy is to use a comparative morphology approach between closely related proteranthous and seranthous species. The approach was applied in the insect-pollinated dogwoods (Genus: \textit{Cornus}) which show a diversity of floral-leaf sequences within the genus \citep{Gunatilleke1984}. While the only three species were compared, the authors found evidence for a tradeoff between pollination efficiency and floral investment with proteranthous flowering \textit{Cornus mas} showing a reduced floral diameter and peduncle length when compared to synthanous \textit{Cornus florida} and seranthous \textit{Cornus sericea}. There are now techniques to quantify investment in plant tissues and trace source-sink carbon dynamics throughout plant organs, and with such tools, these kind of comparative studies applied more broadly in clades with diverging floral-foliate sequences would aid our understanding of proteranthy tremendously.
\par Though the data pertaining to proteranthy are sparse, there seems to be reasonable support for the hypothesis that proteranthous flowering, and other floral-foliate phenological patterns, represent a tradeoff between pollination efficiency and investment. However, there are many fundamental questions about the evolution of proteranthy that need to be addressed: How common is is flowering behavior? Is it correlated with other life history traits? Which models of evolution best describe its distribution? Additionally, to test a hypothesis that is over a century old, researchers should seek to empirically demonstrate the adaptive benefit of proteranthy in deciduous trees.
\section{Floral and foliate phenologies: independent or constrained?}
\par We now understand that phenological stages (hence: phenophases) are not optimized in a vacuum, but timing depends on both leaf and flower physiology and the functional relationship between them. Climate change is already having dramatic impact on tree phenology, and marked shifts towards earlier in spring phenology have been observed \citep{Wolkovich2012}. Will flowering and leafing phenophases shift relative to each other, maintaining their optimized temporal relationship, or will new patterns emerge?
\par At the heart if this inquiry is the question: to what degree is the timing of one phenophase constrained by the other? A broad observation of temperate forest trees shows a significant degree of correlation. Generally, years of earlier leafing than average also are years with earlier flowering than average \citep{Lechowicz1995}, but because of high seasonal variability and complex ecological interactions, observational studies cannot assess whether these patterns are incidental, a product of independent timing of flowering and leafing, or determinate, a product of biological constraints between phenophases. 
\par As I mentioned above, both floral and foliate phenophases are thought to respond to similar environmental cues. This fact alone could explain the phenotypic correlation between flowering and leaf out timing, perhaps through pleiotropy or some shared genetic pathway for phenology that influences both flowering and leaf out timing. It is important to point out that due to their long generation times, trees have not been historically amenable to genetic research. Most of our knowledge about the genetics responsible for phenological responses to the environment come from studies in the annual mustard \textit{Arabidopsis thaliana}. One study of particular relevance, sought to understand how climate conditions (vernalization) might similarly effect the timing of two discrete life history stages, germination and flowering time \citep{Auge2017}. The authors found that genes in the flowering pathway did indeed pleiotropically regulate seed germination. However, few assays found a consistent response to vernalization between the two stages, and in many cases, the response to vernalization was reversed across life stages. As such, the authors conclude that the vernalization genes regulate the different life stages of \textit{Arabidopsis} with a degree of independence. 
\par To my knowledge the only genomic study of phenology in trees was performed in the proteranthous model species \textit{Populus trichocarpa}. These studies identified one gene, FT1, which induced a flowering response due to vernalization, and another, FT2 gene, which facilitated vegetative growth and autumn dormancy as a response to photoperiod \citep{Glover2014}. While this result is promising because there seem to be apparent homologs of these genes in \textit{Arabidopsis}, it has been suggested that the pathways that regulate the flowering response to environmental signals are poorly conserved, with extreme difference among species and within species \citep{Blackman2017}, and as such, it is not possible to suggest any generalities based on the current body of research in this area.
\par I have previously stated that one possible explanation for the floral-foliate patterns we see today is that they are incidental, produced by independent responses to environmental cues. Within the reaction norms of the current climate space, these patterns may appear to be relatively fixed, but if floral and foliate phenophases are differentially sensitive to different environmental cues, and seasonal temperatures change significantly as they are expected to do in the coming decades, these patterns may be disturbed. There is one case study that would reassure us; a study in two species in the cherry family (genus \textit{Prunus}) found that the offset of timing between floral and vegetative bud break could be explained by differential sensitivities to spring warming, with flowering requiring considerably less forcing than leaf out \citep{Guo2014}. If this is the case, we would expect to see these patterns maintained even in an era of global climate change, however these result may be misleading because the authors did not investigate the role of any other environmental cues, such as photoperiod or vernalization in their study.
\par As I mentioned above, there is tremendous variation in how species, and populations within species, respond to climate cues, so I set out to further test the independence of phenological responses between floral and foliate phenophases. I performed a small pilot analysis, in which growth chambers were used to subject three species of woody, deciduous shrubs to four different temperature and photoperiod treatment combinations and compared the phenological response of flower and leaves. Floral and foliate phenological responses were differentially sensitive to changing combinations environmental cues, and the degree of divergence of these responses varied significantly among species. One species, mountain holly  (\textit{Ilex mucronata}), seemed to maintain the temporal offset between leafing and flowering relatively consistently under different treatment combinations, and both phenophase were primarily influenced by temperature. At the other extreme, floral phenology of beaked hazelnut (\textit{Corylus cornuta}) was most sensitive to photoperiod, while foliate phenology was more sensitive to forcing temperature, which under some conditions resulted in a complete reversal to the floral-foliate sequence (results pictured in the figure below).
\includegraphics[width=16cm,height=7cm] {shrubs_4_csee}\\
These results suggest that floral and foliate phenophases can respond to the environment relatively independently of each other, each one tracking its own climate optimum, but that the degree of independence varies significantly among species. 
\par It is important to pause and reflect that we cannot yet judge the adaptive significance of having independent or constrained floral and foliate phenophases in an era of climate change. We can explore hypothetical scenarios to illustrate this uncertainty. Consider proteranthous species with independent floral and foliate phenophases. This independence may allow each phenophase to be expressed at its own climatic optimum, but if, as in the case of beaked hazelnut, leafing is advanced by warming and flowering controlled by photoperiod, the overall impact will be a reduction in the duration of the leafless flowering period. If this state is, as we have explore above, critical for successful pollination in wind pollinated species, the overall impact of climate change in such taxa would be decreased reproductive success, which would ultimately have negative demographic consequences. But there may also be downsides to constrained phenophase expression. If plants are phenologically tracking a warmer climate, proteranthous flowering would be pushed increasingly earlier into a less stable climate space where frost events are more common. If tree flowering were to more often coincide with frost events, this too could reduce the overall fitness of an individual. Yet still, species who do not phenologically track climate change at all will not benefit from an extended growing season, which could put them at a competitive disadvantage to other species in the forest community that are phenologically tracking climate change. We cannot yet predict the likelihood of these scenarios, and just like all other aspects of community dynamics, the boundaries are likely to be fuzzy. However it is clear, that alterations to phenological patterns may significantly impact community dynamics in an era of global climate change.
\section*{Conclusion}
In this paper, I have begun the important work of investigating the relationship between floral and foliate phenophases, which have historically been treated separately in the phenological literature. I discussed the possible importance of floral-foliate phenological patterns in the reproductive success of trees. There is indirect evidence from evolutionary correlations, pollen capture studies and comparative anatomy that the diversity of floral-foliate patterns in trees embody a tradeoff between pollination efficiency and reproductive investment, and can be viewed in the context of classical life history theory as being optimized to the physiological character of a species and the environmental conditions in which they live. I have presented evidence from the literature and my own preliminary work that floral and foliate phenophases can, for many species, respond to the environment relatively independent of one another, implying that the internal phenological patterns we see in individuals have potential to shift considerably under global climate change. I have also discussed the hypothetical changes to forest community dynamics that could be byproducts of alternations to the floral-foliate phenological sequence. The support for the hypothesis that phenological patterns are crucial to the reproductive success and therefore longevity of tree populations, and preliminary evidence that climate change will disrupt these patterns is a call to scientists to better understand the adaptive significance and biological mechanisms that determine floral-foliate phenological patterns, so we can better predict the effects of climate change on forest systems and further develop sustainable forest management techniques in the Anthropocene.


\bibliography{..//refs/oeb53}
\end{document}
